\section{PyQt}

\subsection{¿C\'omo estructurar los proyectos?}
\label{estructura_programas}
Los proyectos que involucran el dise\~no de una GUI utilizando PyQt5 estar\'an compuestos por el Backend y el Frontend, conceptos
que vienen del desarrollo web, y podes ver \href{https://platzi.com/blog/que-es-frontend-y-backend/}{ac\'a}. Entonces, tu proyecto va a tener
c\'odigo en Python de la GUI, de la l\'ogica de lo que hagas, y otros archivos como recursos, stylesheets, imagenes, etc... se propone la siguiente
estructura de archivos muy utilizada en PyQt5, y Python en algunos casos:

\begin{center}
    \begin{minipage}{10cm}
        \dirtree{%
        .1 /project.
        .2 /designer.
        .3 {mainwindow.ui}.
        .2 /resources.
        .3 {resource.qrc}.
        .3 {logo.png}.
        .2 /tests.
        .3 {test\textunderscore backend.py}.
        .2 {/src}.
        .3 {\textunderscore\textunderscore init\textunderscore \textunderscore.py}.
        .3 {app.py}.
        .3 {mainwindow.py}.
        .3 {/ui}.
        .4 {mainwindow.py}.
        .3 {/resources}.
        .4 {resources.py}.
        .2 {main.py}.
        .2 {test.py}.
        .2 requirements.
        .2 readme.
        }
    \end{minipage}
\end{center}

\subsection{¿Qu\'e son Signals, Slots y Property?}
\label{signal_slot_property}

\subsection{¿Qu\'e son los Stylesheets?}
\label{qt_stylesheets}